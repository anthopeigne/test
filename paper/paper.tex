\documentclass[11pt,a4paper]{article}
\usepackage[utf8]{inputenc}
\usepackage{amsmath}
\usepackage{amsfonts}
\usepackage{amssymb}
\usepackage{amsthm}
\usepackage{hyperref}
\usepackage{graphicx}

\theoremstyle{thmstyleone}
\newtheorem{lemma}{Lemma}

\theoremstyle{thmstylethree}
\newtheorem{definition}{Definition}



%%%%%%%%%%%%%%%%%%%%%
%%Coq documentation%%
%%%%%%%%%%%%%%%%%%%%%
\def\BaseUrl{https://anthopeigne.github.io/test/}
\newcommand{\coqdoc}[1]{\href{\BaseUrl/#1}{\raisebox{-.9mm}{\includegraphics[height=1em]{coql.png}}}}


\begin{document}

\section{Boolean stuff}

Boole is cool.

\begin{definition}[NXOR]
	\coqdoc{More.html\#nxorb}
	We define the NXOR boolean operator that we denote as $\odot$ by the truth table
	\[
		\begin{array}{|c|c|c|}
			\hline
			p & q & p \odot q \\
			\hline
			F & F & T \\
			\hline
			F & T & F \\
			\hline
			T & F & F \\
			\hline
			F & F & T \\
			\hline
		\end{array}
	\]
\end{definition}

\begin{lemma}
	\coqdoc{More.html\#nxorb\_invol}
	$(p \odot q) \odot q = p$.
\end{lemma}
\end{document}